\chapter{Data acquisition}
\chapterauthor{Sean Goggins}
\section{Introduction}
\paragraph{} 
\paragraph{If you give a mouse a cookie, she is going to ask you for a glass of milk}\cite{numeroff_if_1985}. Similarly, if one provides an analyst, stakeholder or researcher with data representing one aspect, section or dimension of a phenomena, they are going to ask for a glass of milk. Or additional perspectives. This is the experience of having too much data, and possibly too much access to data. 
\paragraph{} Data acquisition in the land of plenty has two contrasting cases; or perhaps we say that data scarcity comes in two forms. Scarcity can include cases where there are not available repositories of research data. For example, while there is great interest in examining diversity and inclusion in open source software, our usual software repositories do not capture essential demographic information. Scarcity can also be related to access. Perhaps there is a privileged data repository that you would like access to, but are unable to obtain. In this case you are "locked out" \cite{goggins_group_2013}. 
\paragraph{The Mining Software Respositories, open collaboration data exchange and GenBank communities provide examples open source software ecosystem researchers may be able to utilize as exemplars to emulate for tracking data provenance.}  Data provenance is a record of how data was collected, processed, altered and reshaped to provide the foundation for analysis and findings reported in academic papers or professional, industrial project dashboards. \#TODO describe prior work looking at data provenance across fields
\paragraph{Reporting of the provenance of data used for research and practice examining open source software is inconsistent.} \#\#TODO Describe examples and make the gap clear
\paragraph{In this chapter we enumerate the challenges, solutions and 2 exemplars for managing and documenting data provenance in open source software research.}
\section{State of the Art}

\section{Challenges in Research Practice}

\section{Methods for Addressing Challenges}

\section{The Open Collaboration Data Exchange Manifest Structure}
\subsection{The SPDX Example in the Linux Foundation}
\subsection{OCDX Current State}

\section{Data Provenance Technical Pipeline Examples From Social Computing Research}

\section{Data Provenance Technical Pipeline Examples From The Mining Software Repositories Community}


\section{Next Steps}

\section{Conclusion}